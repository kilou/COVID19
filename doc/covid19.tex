\documentclass[10pt,twocolumn,a4paper]{article}

\usepackage[utf8]{inputenc}
\usepackage[T1]{fontenc}
\usepackage{lmodern}
\usepackage{amsmath}
\usepackage{amsthm}
\usepackage{amsfonts}
\usepackage{amssymb}
\usepackage[left=1.8cm,right=1.8cm,top=2.5cm,bottom=2.5cm]{geometry}
%\usepackage{bibtex}

% --------------------------------------------------------------------------- %

% Independence symbol
\newcommand\independent{\protect\mathpalette{\protect\independenT}{\perp}}
\def\independenT#1#2{\mathrel{\rlap{$#1#2$}\mkern2mu{#1#2}}}

% Negative Binomial symbol
\newcommand{\NB}{\text{NegBin}}

% Numbered environments
\theoremstyle{definition}
\newtheorem{definition}{Definition}
\newtheorem{hypothesis}{Hypothesis}
\theoremstyle{remark}
\newtheorem{remark}{Remark}
\theoremstyle{plain}
\newtheorem{lemma}{Lemma}

% --------------------------------------------------------------------------- %

\author{
  A. Chaouch, Y. Eggli, J. Pasquier, V. Rousson and B. Trächsel\\
  Center for Primary Care and Public Health\\
  University of Lausanne (Switzerland)
}

\title{
  Covid19 - Forecasts ICU beds
}

%\date{\today}
\date{April 2020}

% --------------------------------------------------------------------------- %

\begin{document}

\maketitle

\begin{definition}

  Let $Y_k$ be the cumulative count of hospitalized patients on day $k$.
  %
  \[ Y_k = (1 + \exp(u_k)) Y_{k-1}, \]
  %
  where
  %
  \[ u_k \sim \text{N}(\log(\lambda_k-1), \sigma_k^2). \]
  %
  Note: $\lambda_k$ = \verb|mlam| and $\sigma_k^2$ = \verb|vlam|.

\end{definition}

A voir : $\mu_k \independent Y_{k-1} \implies Y_k \sim \text{Log-normal}$

\begin{definition}

  Let $p_k$ be the proportion of patients hospitalized on day $k$ (incident
  cases) who will then be admitted to intensive care. The random variable $p_k$
  follows a logit-normal distribution:
  %
  \[ p_k = \cfrac{\exp(q_k)}{1 + \exp(q_k)}, \]
  %
  where
  %
  \[ q_k \sim \text{N}(\log(\cfrac{\pi_k}{1-\pi_k}), \tau_k^2). \]
  %
  Note: $\pi_k$ = \verb|mpic| and $\tau_k^2$ = \verb|vpic|.

\end{definition}

\begin{definition}

  Let $S_k$ be the number of days (possibly zero) between the hospitalization
  and the admission in the intensive care for the patients who were
  hospitalized on day $k$.
  %
  \[ S_k \sim \NB(\cfrac{\phi_k^2-\mu_k}{\phi_k^2},
                  \cfrac{\mu_k^2}{\phi_k^2-\mu_k}). \]
  %
  Note: $\mu_k$ = \verb|mlag| and $\phi_k^2$ = \verb|vlag|.

\end{definition}

\begin{definition}

  Let $T_k$ the length of stay (possibly zero) in the intensive care for the
  patients who were hospitalized on day $k$.
  %
  \[ T_k \sim \NB(\cfrac{\chi_k^2-\nu_k}{\chi_k^2},
                  \cfrac{\nu_k^2}{\chi_k^2-\nu_k}). \]
  %
  Note: $\nu_k$ = \verb|mlos| and $\chi_k^2$ = \verb|vlos|.

\end{definition}

\begin{definition}

  Let define the following random variable:
  %
  \[
    I_{j,k,l} =
    \begin{cases}
      1 & \parbox[t]{.3\textwidth}{if the $j$th patient hospitalized on day $k$
          is being treated in intensive care $l$ day later,} \\
      0 & \text {if not.}
    \end{cases}
  \]

\end{definition}

\begin{lemma}

  $P(I_{j,k,l}=1) = \sum\limits_{i=0}^l P(S_k = i, T_k \ge l-i)$

\end{lemma}

\begin{proof}

  \begin{align*}
    P(I_{j,k,l}=1) &= P(S_k \le l \le S_k + T_k) \\
                   &= \sum\limits_{i=0}^l P(S_k = i, T_k \ge l-i)
  \end{align*}

\end{proof}

\begin{definition}

  Let $Z_m$ the number of patients treated in intensive care on day $m$.

\end{definition}

\begin{lemma}

  \[ Z_m = \sum\limits_{k=1}^m \sum\limits_{j=1}^{W_k} I_{j,k,m-k} \]
  %
  where $W_k = p_k (Y_k - Y_{k-1})$ with the convention that $Y_0 = 0$.

\end{lemma}


\begin{lemma}

  Suppose that the random variables \dots are independent, then
  %
  \[ E(Z_m) \approx \dots \]

\end{lemma}

\begin{proof}

  \begin{align*}
    E(Z_m) &= E(\sum_{k=1}^m W_k I_{j,k,m-k}) \\
           &= \sum_{k=1}^m E(p_k) (E(Y_k) - E(Y_{k-1})) E(I_{j,k,m-k}) \\
    E(p_k) &\approx \cfrac{1}{K-1} \sum_{i=1}^{K-1}
                    \text{logit}(\Phi_{.,.}^{-1}(i/K)) \\
    E(Y_k) &= (1 + \exp(log(\lambda_k-1)+\sigma_k^2/2) E(Y_{k-1}) \\
    E(I_{j,k,m-k}) &= \sum_{i=0}^l P(S_k = i, T_k \ge l-i) \\
                   &= \sum_{i=0}^l P(S_k = i) P(T_k \ge l-i)
  \end{align*}

\end{proof}


\end{document}
