%\documentclass[t,9pt,xcolor=dvipsnames,notes]{beamer}        % print frame + notes
%\documentclass[t,9pt,xcolor=dvipsnames,notes=only]{beamer}   % only notes
\documentclass[t,9pt,xcolor=dvipsnames]{beamer}              % only frames
%\documentclass[t,9pt,xcolor=dvipsnames,handout,notes=show]{beamer}      % print frame + notes (no pause)
%\documentclass[t,9pt,xcolor=dvipsnames,handout]{beamer}      % print frame only (no pause)

% Metropolis theme
\usetheme[outer/progressbar=frametitle, sectionpage=progressbar, subsectionpage=none, numbering=fraction, block=fill]{metropolis}
\makeatletter
\setlength{\metropolis@titleseparator@linewidth}{0.5pt}
\setlength{\metropolis@progressonsectionpage@linewidth}{0.5pt}
\setlength{\metropolis@progressinheadfoot@linewidth}{0.5pt}
\def\@fnsymbol#1{\ensuremath{\ifcase#1\or \dagger\or \ddagger\or
   \mathsection\or \mathparagraph\or \|\or **\or \dagger\dagger
   \or \ddagger\ddagger \else\@ctrerr\fi}}
\makeatother

% Bibliography style
%\usepackage[style=authoryear,autocite=footnote,backend=bibtex,isbn=false,url=false]{biblatex}
% \addbibresource{../Thesis/Bibliography.bib}
\usepackage{xpatch}
\xapptobibmacro{cite}{\setunit{\nametitledelim}\printfield{title}\printfield{journal}}{}{}
\usepackage{tabularx}

% Additional packages
%\usepackage[T1]{fontenc}
%\usepackage[sfdefault,scaled=.85]{FiraSans}
%\usepackage{mathpazo}
%\usepackage{eulervm}
%\usepackage{newtxsf} 

\usepackage{amsmath,amsfonts,amssymb,amscd,amsthm,xspace,bm,cases,etoolbox}
\usepackage{xcolor} 
\usepackage{perpage}
\usepackage{graphics}
\usepackage{tikz}
\usetikzlibrary{shapes,arrows,positioning}
\usepackage{ragged2e}
\usepackage{hyperref}

\AtBeginEnvironment{pmatrix}{\setlength{\arraycolsep}{2pt}} % narrower columns in pmatrix (requires etoolbox package)

% Specify other colors and options as required
%\setbeamercolor{alerted text}{fg=Maroon}
%\setbeamertemplate{items}[square]
\setbeamertemplate{blocks}[rounded]
\beamertemplatenavigationsymbolsempty
%\setbeamercovered{transparent=20}
\setbeamerfont{alerted text}{series=\bfseries}

\MakePerPage{footnote}
\renewcommand{\footnotesize}{\tiny} 
\renewcommand{\thefootnote}{\fnsymbol{footnote}}

\setbeamertemplate{navigation symbols}{\usebeamercolor[fg]{title in head/foot}\usebeamerfont{title in head/foot}\insertframenumber/\inserttotalframenumber}
\setbeamertemplate{footline}{}

\DeclareMathAlphabet{\mathpzc}{OT1}{pzc}{m}{it}

\newlength{\textlarg}
\newcommand{\barre}[1]{%
   \settowidth{\textlarg}{#1}
   #1\hspace{-\textlarg}\rule[0.5ex]{\textlarg}{0.5pt}}

\newcommand{\backupbegin}{
   \newcounter{framenumberappendix}
   \setcounter{framenumberappendix}{\value{framenumber}}
}
\newcommand{\backupend}{
   \addtocounter{framenumberappendix}{-\value{framenumber}}
   \addtocounter{framenumber}{\value{framenumberappendix}} 
}
  

\newcommand{\argmax}{\arg\!\max}
\newcommand\numberthis{\addtocounter{equation}{1}\tag{\theequation}}

\newcommand{\argmin}{\mathop{\mathrm{argmin}}\limits}

\newcommand{\e}[1]{ \ensuremath{\text{E}\!\left[{#1}\right]} }

%\setbeamerfont{institute}{size=\fontsize{6pt}{7pt}}

%\renewcommand\thempfootnote{\arabic{mpfootnote}}

% Title and author information
\title{PROJECTIONS OF ICU OCCUPATION AND HOSPITAL MORTALITY DURING THE COVID-19 EPIDEMIC IN THE CANTON OF VD}
\subtitle{}
\author{Aziz Chaouch, Yves Eggli, Isabella Locatelli, Jérôme Pasquier, Valentin Rousson, Bastien Trächsel}
\institute{Center for Primary Care and Public Health (Unisanté), University of Lausanne, Switzerland}
\date{May, 2020}

\begin{document}

% ----------------------------------------------------------------------------------
\begin{frame}
\titlepage

% ***** Comments
\note{
\tiny

}
\end{frame}

% ----------------------------------------------------------------------------------
\begin{frame}{Mandate}
\justifying

In march 2020, DGS (Direction Générale de la Santé) contacted us to develop a software tool to follow COVID-19 epidemic in the Canton of VD and project the following quantities under different scenarios:
\begin{enumerate}
\item Global bed occupation in intensive care units (ICU) in VD hospitals (initial request)
\item Hospital mortality (additional request)
\end{enumerate}
\pause

\alert{Goals}: 
\begin{itemize}
\item Assess whether and when ICUs could be  overflooded by COVID-19 patients ($\sim$ 200 ICUs beds for COVID-patients available in VD)
\item Assess whether ice skating rinks would be needed to store deceased patients before funerals
\end{itemize}
\medskip

In the event of ICUs overflooding, patients admission to ICU would be restricted and granted only to those patients with the best chances of survival. A patient in need of IC and whose access to ICU is refused is virtually condemned to die.

% ****** Comments
\note{
\tiny
\justifying

}
\end{frame}

% ----------------------------------------------------------------------------------
\begin{frame}{Data}
\justifying

DGS initially provided \alert{aggregated data} with daily information on
\begin{itemize}
\item cumulative counts of COVID-19 hospitalized patients
\item number of ICU beds occupied by COVID-19 patients
\item number of COVID-19 related deaths
\end{itemize}
\bigskip

and later provided anonymized \alert{individual patient data} with
\begin{itemize}
\item age, sex
\item dates of hospital admission/discharge and ICU transfer/discharge (if applicable)
\item health status (alive/dead) on hospital discharge
\end{itemize}

Data (Excel file) was updated every day or couple of days. Last available update on April 17, 2020.

% ****** Comments
\note{
\tiny
\justifying

}
\end{frame}


% ##################################################################################
\section{Projections for ICU beds occupation}

% ----------------------------------------------------------------------------------
\begin{frame}{Projections for ICU beds occupation}

Based on daily \alert{cumulative number of hospitalizations} (more precise than cumulative counts of detected cases which depend on screening policy)
\medskip

Projections based on 5 potentially time-varying parameters:

\begin{enumerate}
\item \alert{EGP} = Exponential Growth Parameter for cumulative number of hospitalizations
\bigskip

\item \alert{ICP} = Intensive Care Proportion (\% hospitalized patients that will require IC at some point)
\bigskip

\item \alert{LAG} = Time interval between hospital admission and ICU transfer
\bigskip

\item \alert{ADP} = Admission probability (\% patient requiring IC that will be admitted in ICUs)
\bigskip

\item \alert{LOS} = Length of Stay in ICU
\end{enumerate}

Each parameter drawn from a distribution with potentially time-varying mean/median and variance. User input crucial for EGP and ADP!

% ****** Comments
\note{
\tiny
\justifying
Unlike the number of infected cases, counts of hospitalized patients are unaffected by the screening policy
}
\end{frame}

% ----------------------------------------------------------------------------------
\begin{frame}{1. EGP: Exponential Growth Parameter}

Defines how the \alert{cumulative number of hospitalized patients} grows from day $j$ to day $j+1$

Let $N_{j}$ denote the cumulative number of hospitalized patients up to day $j$. The cumulative number of hospitalized patient on day $j+1$ is  given by
\begin{equation*}
N_{j+1}= \text{EGP}_{j+1} \cdot N_{j}
\end{equation*}

with $\text{EGP}_{j} \geq 1$ and $\text{EGP}_{j}=1$ defining the end of the epidemic.
\medskip

Distribution:
\begin{equation*}
\log(\text{EGP}_{j}-1) \sim \mathcal{N}\left(\log(\text{megp}_{j}-1), \text{vegp}_{j}\right)
\end{equation*}
\medskip

Projected cumulative counts of hospitalizations were used to derive daily new hospitalizations (i.e. incident cases).

% ****** Comments
\note{
\tiny
\justifying

}
\end{frame}

% ----------------------------------------------------------------------------------
\begin{frame}{2. ICP: Intensive Care Proportion}

Defines proportion of hospitalized patients that will require IC at some point. DGS estimated that 1 hospitalized patient out of 5 would require IC (verified on individual patient data).
\medskip

Distribution:
\begin{equation*}
\text{logit}(\text{ICP}_{j}) \sim \mathcal{N}\left(\text{logit}(\text{micp}_{j}), \text{vicp}_{j}\right)
\end{equation*}
\medskip

Simulation of IC requirement for a patient:
Let $g_{ij}$ denote the indicator variable for IC requirement for patient $i$ among patients who were admitted to hospital on day $j$, such that
\begin{itemize}
\item $g_{ij}=1$: patient $i$ will require IC at some point (during hospitalization)
\item $g_{ij}=0$: patient $i$ will not require IC
\end{itemize}

Then $g_{ij}$ is simulated according to
\begin{equation*}
g_{ij} \sim \text{Bernoulli}\left(\text{ICP}_{j}\right)
\end{equation*} 

% ****** Comments
\note{
\tiny
\justifying

}
\end{frame}

% ----------------------------------------------------------------------------------
\begin{frame}{3. LAG: Time to (theoretical) ICU transfer}

Defines number of days between hospital admission and ICU transfer (only applies to patients requiring IC (i.e. $g_{ij}=1$)
\medskip

Distribution:
\begin{equation*}
\text{LAG}_{ij} \sim \text{NegBin}\left(\text{mlag}_{j}, \text{vlag}_{j}\right)
\end{equation*}
\medskip

Estimates based on available data: mlag=2 and vlag=9 days ($q_{95\%}=8, q_{99\%}=14$)

% ****** Comments
\note{
\tiny
\justifying

}
\end{frame}

% ----------------------------------------------------------------------------------
\begin{frame}{4. ADP: (ICU) Admission Probability}

Defines proportion of patients with $g_{ij}=1$ that will be effectively admitted in ICU on day $k=j+\text{LAG}_{ij}$. Ideally $\text{ADP}_{j}=1$ (no restriction).
\medskip

Distribution (if $\text{ADP}_{j}<1$):
\begin{equation*}
\text{logit}(\text{ADP}_{j}) \sim \mathcal{N}\left(\text{logit}(\text{madp}_{j}), \text{vadp}_{j}\right)
\end{equation*}

Ideal situation: \

Simulation of admittance of patient $i$ on day $k$:
Let $a_{ij}$ denote the indicator variable for ICU admission for patient $i$ (with $g_{ij}=1$) on day $k=j+\text{LAG}_{ij}$, such that
\begin{itemize}
\item $a_{ij}=1$: patient $i$ is admitted in ICU on day $k$
\item $a_{ij}=0$: patient $i$ is not admitted in ICU on day $k$ (=death)
\end{itemize}

Then $a_{ij}$ is simulated according to
\begin{equation*}
a_{ij} \sim \text{Bernoulli}\left(\text{ADP}_{j}\right)
\end{equation*} 

% ****** Comments
\note{
\tiny
\justifying

}
\end{frame}



% ----------------------------------------------------------------------------------
\begin{frame}{5. LOS: Length of Stay in ICU}

Defines length of stay (nb days) in ICU (only applies to patients requiring IC)
\medskip

Distribution:
\begin{equation*}
\text{LOS}_{ij} \sim \text{NegBin}\left(\text{mlos}_{j}, \text{vlos}_{j}\right)
\end{equation*}

Estimates based on available data: mlos=13 and vlos=154 days ($q_{95\%}=38, q_{99\%}=57$)

% ****** Comments
\note{
\tiny
\justifying

}
\end{frame}

% ----------------------------------------------------------------------------------
\begin{frame}{Simulation steps}

\begin{enumerate}
\item Draw parameters $\text{EGP}_{j}$, $\text{ICP}_{j}$ and $\text{ADP}_{j}$ from their distribution for each day $j$
\item Project cumulative counts of hospitalizations using EGP and derive incident hospitalized cases based on observed/projected cumulative counts
\item Generate binary indicator $g_{ij} \sim \text{Bernoulli}(\text{ICP}_{j})$ for fake patient $i$ hospitalized on day $j$
\item Restrict attention to those patients with simulated $g_{ij}=1$ and generate $\text{LAG}_{ij} \sim \text{NegBin}(\text{mlag}_{j},\text{vlag}_{j})$
\item Decide whether patient $i$ will be admitted in ICU on day $k=j+\text{LAG}_{ij}$ by generating $a_{ij} \sim \text{Bernoulli}(\text{ADP}_{j})$
\item For patients with $a_{ij}=1$, generate $\text{LOS}_{ij} \sim \text{NegBin}(\text{mlos}_{j},\text{vlos}_{j})$
\item Build bed occupancy matrix and sum-up nb of beds occupied each day
\end{enumerate}

Simulations repeated 1000 times (could be increased...) $\rightarrow$ construct \alert{predictive distribution of the number of occupied ICU beds on each day} (from start of the epidemic until $X$ days in the future). Predictive distribution summarized using median and e.g. quantiles 5\% and 95\%

% ****** Comments
\note{
\tiny
\justifying

}
\end{frame}

% ##################################################################################
\section{Projections for hospital mortality}
\note{
\tiny
\justifying

}

% ----------------------------------------------------------------------------------
\begin{frame}{Projections for hospital mortality}
\justifying

Uses daily number of new hospitalizations (as observed or projected using EGP)

We wanted mortality to be adjusted for age (strong predictor) and possibly sex/IC requirements. The following age groups were defined:
\medskip
\begin{table}[ht]
\centering
 \begin{tabular}{|c|c|c|} \hline
 Age group & Global proportion & Proportion of females \\
 \hline
 $<$ 70 y & 50\% & 37\% \\
 70-85 y & 33\% & 41\% \\
 $\geq$ 85 y & 17\% & 53\%\\
 \hline
 \end{tabular}
 \end{table}
 \medskip
 
Simple broad age categories enables investigation of scenarios were the structure of the population of hospitalized patients changes with time (e.g. future arrival of old patients from nursing homes etc)

For this reason, ICP (disease severity), LAG and total hospital LOS were derived using statistical models and were not allowed to vary over time anymore!

% ****** Comments
\note{
\tiny
\justifying

}
\end{frame}

% ----------------------------------------------------------------------------------
\begin{frame}{Probability to require IC and lag}
\justifying

The probability $p_{i}$ that patient $i$ did require IC at some point during hospitalization was modeled as a function of age and sex in a logistic regression model:

\begin{equation*}
\text{logit}(p_{i})=f(\text{age}_{i}+\text{sex}_{i})
\end{equation*}

IC indicator variable for new patients simulated as $g_{i} \sim \text{Bernoulli}(p_{i})$

For existing patients without IC information (i.e. those still in hospital but not yet admitted in ICU), the indicator $g_{i}$ was imputed  using this model ($M=50$ imputations).

Lag was generated according to $\text{LAG}_{i} \sim \text{Weibull}(\mu,\sigma)$

% ****** Comments
\note{
\tiny
\justifying

}
\end{frame}

% ----------------------------------------------------------------------------------
\begin{frame}{Survival models for competitive risks}
\justifying

\alert{Time to death}

Event: death

Censoring: patient still in hospital + patient who left hospital alive (i.e. cured)
\begin{align*}
T_{1i} & \sim \text{Weibull}(\mu_{1i},\sigma_{1}) \\
\log(\mu_{1i}) & = f_{1}(\text{age}_{i}+\text{sex}_{i}+g_{i}) \\
h_{1i} & = \text{instant hazard}
\end{align*}

\alert{Time to hospital release (dead or alive)}

Event: hospital release (dead or alive)

Censoring: patient still in hospital
\begin{align*}
T_{2i} & \sim \text{Weibull}(\mu_{2i},\sigma_{2}) \\
\log(\mu_{2i}) & = f_{2}(\text{age}_{i}*g_{i}+\text{sex}_{i}) \\
h_{2i} & = \text{instant hazard}
\end{align*}

Probability of dying at the end of hospital given by $h_{1i}/h_{2i}$


% ****** Comments
\note{
\tiny
\justifying

}
\end{frame}

% ----------------------------------------------------------------------------------
\begin{frame}{Simulating patient death (in hospital)}
\justifying

Let $n_{j}$ define the number of existing/projected new hospitalizations on day $j$:
\begin{enumerate}
\item Draw age/sex from defined distribution (i.e. age categories)
\item Generate IC status $g_{i}$ according to logistic regression model for $i=1,...,n_{j}$
\item Generate total hospital LOS $T_{2i}$ using Weibull model for time to hospital release (adjusted for age, sex and $g_{i}$)
\item If $g_{i}=1$, generate a lag time according to lag model (conditionned on $\text{LAG}_{i}<=\text{LOS}_{i}$)
\item Calculate hazard of death $h_{1i}=h_{1}(T_{2i})$ and hazard of exiting hospital either alive or dead $h_{2i}=h_{2}(T_{2i})$
\item Draw death indicator $\delta_{i} \sim \text{Bernoulli}(h_{1i}/h_{2i})$
\end{enumerate}

1000 simulations: allows to construct predictive distribution of the cumulative number of deaths occurring every day. Predictive distribution summarized using median and e.g. quantiles 5\% and 95\%

% ****** Comments
\note{
\tiny
\justifying

}
\end{frame}

% ----------------------------------------------------------------------------------
\begin{frame}{Three types of mortality projections}
\justifying

\alert{Type 3}
Require aggregated data on hospitalizations (mortality data optional). Uses observed cumulative counts of hospitalized patients and generate new patients/deaths from the start of the epidemic
\begin{itemize}
\item Prediction intervals for past. Can be compared to observed cumulative death counts (when available) for "goodness of fit"
\item Mimics what could have happened in a similar epidemic with same cumulative number of hospitalizations but with different patients
\end{itemize}
\pause

\alert{Type 2}
Require aggregated data on hospitalizations + cumulative death counts. Same as for type 1 except that simulated deaths in the past are simply replaced with observed death counts in all simulations
\begin{itemize}
\item No prediction intervals for past
\item Only an approximation!
\end{itemize}
\pause

\alert{Type 1}
Require individual patient data. Uses all individual patient data (i.e. observed age, sex, IC and death status)
\begin{itemize}
\item No prediction intervals for past (as for type 2)
\item Coherent deaths counts (i.e. accounts for all available information)
\end{itemize}


% ****** Comments
\note{
\tiny
\justifying

}
\end{frame}

% ----------------------------------------------------------------------------------
\begin{frame}{Software links}
\justifying

\bigskip
R Shiny application available on dedicated server (at least up to June 5, 2020):
\bigskip

\begin{center}
\href{https://stat-cmb.ddns.net/COVID19/}{\color{blue}{\underline{https://stat-cmb.ddns.net/COVID19/}}}
\end{center}

Source code: 
\begin{center}
\href{https://github.com/kilou/COVID19}{\color{blue}{\underline{https://github.com/kilou/COVID19}}}
\end{center}


% ****** Comments
\note{
\tiny
\justifying

}
\end{frame}

\end{document}
