%\documentclass[t,9pt,xcolor=dvipsnames,notes]{beamer}        % print frame + notes
%\documentclass[t,9pt,xcolor=dvipsnames,notes=only]{beamer}   % only notes
\documentclass[t,9pt,xcolor=dvipsnames]{beamer}              % only frames
%\documentclass[t,9pt,xcolor=dvipsnames,handout,notes=show]{beamer}      % print frame + notes (no pause)
%\documentclass[t,9pt,xcolor=dvipsnames,handout]{beamer}      % print frame only (no pause)

% Metropolis theme
\usetheme[outer/progressbar=frametitle, sectionpage=progressbar, subsectionpage=none, numbering=fraction, block=fill]{metropolis}
\makeatletter
\setlength{\metropolis@titleseparator@linewidth}{0.5pt}
\setlength{\metropolis@progressonsectionpage@linewidth}{0.5pt}
\setlength{\metropolis@progressinheadfoot@linewidth}{0.5pt}
\def\@fnsymbol#1{\ensuremath{\ifcase#1\or \dagger\or \ddagger\or
   \mathsection\or \mathparagraph\or \|\or **\or \dagger\dagger
   \or \ddagger\ddagger \else\@ctrerr\fi}}
\makeatother

% Bibliography style
%\usepackage[style=authoryear,autocite=footnote,backend=bibtex,isbn=false,url=false]{biblatex}
% \addbibresource{../Thesis/Bibliography.bib}
\usepackage{xpatch}
\xapptobibmacro{cite}{\setunit{\nametitledelim}\printfield{title}\printfield{journal}}{}{}
\usepackage{tabularx}

% Additional packages
%\usepackage[T1]{fontenc}
%\usepackage[sfdefault,scaled=.85]{FiraSans}
%\usepackage{mathpazo}
%\usepackage{eulervm}
%\usepackage{newtxsf} 

\usepackage{amsmath,amsfonts,amssymb,amscd,amsthm,xspace,bm,cases,etoolbox}
\usepackage{xcolor} 
\usepackage{perpage}
\usepackage{graphics}
\usepackage{tikz}
\usetikzlibrary{shapes,arrows,positioning}
\usepackage{ragged2e}

\AtBeginEnvironment{pmatrix}{\setlength{\arraycolsep}{2pt}} % narrower columns in pmatrix (requires etoolbox package)

% Specify other colors and options as required
%\setbeamercolor{alerted text}{fg=Maroon}
%\setbeamertemplate{items}[square]
\setbeamertemplate{blocks}[rounded]
\beamertemplatenavigationsymbolsempty
%\setbeamercovered{transparent=20}
\setbeamerfont{alerted text}{series=\bfseries}

\MakePerPage{footnote}
\renewcommand{\footnotesize}{\tiny} 
\renewcommand{\thefootnote}{\fnsymbol{footnote}}

\setbeamertemplate{navigation symbols}{\usebeamercolor[fg]{title in head/foot}\usebeamerfont{title in head/foot}\insertframenumber/\inserttotalframenumber}
\setbeamertemplate{footline}{}

\DeclareMathAlphabet{\mathpzc}{OT1}{pzc}{m}{it}

\newlength{\textlarg}
\newcommand{\barre}[1]{%
   \settowidth{\textlarg}{#1}
   #1\hspace{-\textlarg}\rule[0.5ex]{\textlarg}{0.5pt}}

\newcommand{\backupbegin}{
   \newcounter{framenumberappendix}
   \setcounter{framenumberappendix}{\value{framenumber}}
}
\newcommand{\backupend}{
   \addtocounter{framenumberappendix}{-\value{framenumber}}
   \addtocounter{framenumber}{\value{framenumberappendix}} 
}
  

\newcommand{\argmax}{\arg\!\max}
\newcommand\numberthis{\addtocounter{equation}{1}\tag{\theequation}}

\newcommand{\argmin}{\mathop{\mathrm{argmin}}\limits}

\newcommand{\e}[1]{ \ensuremath{\text{E}\!\left[{#1}\right]} }

%\setbeamerfont{institute}{size=\fontsize{6pt}{7pt}}

%\renewcommand\thempfootnote{\arabic{mpfootnote}}

% Title and author information
\title{PROJECTIONS OF ICU OCCUPATION AND HOSPITAL MORTALITY DURING THE COVID-19 EPIDEMIC IN THE CANTON OF VD}
\subtitle{}
\author{Aziz Chaouch, Yves Eggli, Isabella Locatelli, Jérôme Pasquier, Valentin Rousson, Bastien Trächsel}
\institute{Center for Primary Care and Public Health (Unisanté), University of Lausanne, Switzerland}
\date{May, 2020}

\begin{document}

% ----------------------------------------------------------------------------------
\begin{frame}
\titlepage

% ***** Comments
\note{
\tiny

}
\end{frame}

% ----------------------------------------------------------------------------------
\begin{frame}{Mandate}
\justifying

In march 2020, DGS (Direction Générale de la Santé) contacted us to develop a software tool to follow COVID-19 epidemic in the Canton of VD and project the following quantities under different scenarios:
\begin{enumerate}
\item global bed occupation in intensive care units (ICU) in VD hospitals (initial request)
\item Hospital mortality (additional request)
\end{enumerate}

\alert{Goals}: Assess whether ICUs would be  overflooded by patients and take

In the event of ICUs overflooding, patients admission to ICU would be restricted and granted only to those patients with the best chances of survival. A patient in need of IC and whose access to ICU is refused is virtually condemned to die.

% ****** Comments
\note{
\tiny
\justifying

}
\end{frame}

% ----------------------------------------------------------------------------------
\begin{frame}{Data}
\justifying

DGS initially provided \alert{aggregated data} with daily information on
\begin{itemize}
\item cumulative counts of COVID-19 hospitalized patients
\item number of ICU beds occupied by COVID-19 patients
\item number of COVID-19 related deaths
\end{itemize}
\bigskip

and later provided anonymized \alert{individual patient data} with
\begin{itemize}
\item age, sex
\item dates of hospital admission/discharge and ICU transfer/discharge (if applicable)
\item health status (alive/dead) on hospital discharge
\end{itemize}

Data (Excel file) was updated every day or couple of days. Last available update on April 17, 2020.

% ****** Comments
\note{
\tiny
\justifying

}
\end{frame}


% ##################################################################################
\section{Projections for ICU beds occupation}

% ----------------------------------------------------------------------------------
\begin{frame}{Projections for ICU beds occupation}

Based on daily \alert{cumulative number of hospitalizations} and the following 5 parameters:

\begin{enumerate}
\item \alert{EGP} = Exponential Growth Parameter for cumulative number of hospitalizations
\bigskip

\item \alert{ICP} = Intensive Care Proportion (\% hospitalized patients that will require IC at some point)
\bigskip

\item \alert{LAG} = Time interval between hospital admission and ICU transfer
\bigskip

\item \alert{ADP} = Admission probability (\% patient requiring IC that will be admitted in ICUs)
\bigskip

\item \alert{LOS} = Length of Stay in ICU
\end{enumerate}

Each parameter taken from distributions and allowed to vary over time. User input especially import for EGP and ADP.

% ****** Comments
\note{
\tiny
\justifying
Unlike the number of infected cases, counts of hospitalized patients are unaffected by the screening policy
}
\end{frame}

% ----------------------------------------------------------------------------------
\begin{frame}{1. EGP: Exponential Growth Parameter}

Defines how the \alert{cumulative number of hospitalized patients} grows from day $j$ to day $j+1$

Let $N_{j}$ denote the cumulative number of hospitalized patients up to day $j$. The cumulative number of hospitalized patient on day $j+1$ is  given by
\begin{equation*}
N_{j+1}= \text{EGP}_{j+1} \cdot N_{j}
\end{equation*}

with $\text{EGP}_{j} \geq 1$ and $\text{EGP}_{j}=1$ defining the end of the epidemic.
\medskip

Distribution:
\begin{equation*}
\log(\text{EGP}_{j}-1) \sim \mathcal{N}\left(\log(\text{megp}_{j}-1), \text{vegp}_{j}\right)
\end{equation*}
\medskip

Projected cumulative counts of hospitalizations were used to derive daily new hospitalizations (i.e. incident cases).

% ****** Comments
\note{
\tiny
\justifying

}
\end{frame}

% ----------------------------------------------------------------------------------
\begin{frame}{2. ICP: Intensive Care Proportion}

Defines proportion of hospitalized patients that will require IC at some point. DGS estimated that 1 hospitalized patient out of 5 would require IC (verified on individual patient data).
\medskip

Distribution:
\begin{equation*}
\text{logit}(\text{ICP}_{j}) \sim \mathcal{N}\left(\text{logit}(\text{micp}_{j}), \text{vicp}_{j}\right)
\end{equation*}
\medskip

\alert{Among new hospitalized patients on day $j$}, let $g_{ij}$ denote the the need for IC of patient $i$
\begin{itemize}
\item $g_{ij}=1$: patient $i$ will require IC at some point (during hospitalization)
\item $g_{ij}=0$: patient $i$ will not require IC
\end{itemize}

For each new hospitalized patient $i$ on day $j$, simulate $g_{ij}$ as
\begin{equation*}
g_{ij} \sim \text{Bernoulli}\left(\text{ICP}_{j}\right)
\end{equation*} 

% ****** Comments
\note{
\tiny
\justifying

}
\end{frame}

% ----------------------------------------------------------------------------------
\begin{frame}{3. LAG: Time to (theoretical) ICU transfer}

Defines number of days between hospital admission and ICU transfer (only applies to patients requiring IC( i.e. $g_{ij}=1$).
\medskip

Distribution:
\begin{equation*}
\text{LAG}_{ij} \sim \text{NegBin}\left(\text{mlag}_{j}, \text{vlag}_{j}\right)
\end{equation*}
\medskip

Estimates based on available data: mlag=2 and vlag=9 ($q_{5\%}=0$, $q_{95\%}=8$).

% ****** Comments
\note{
\tiny
\justifying

}
\end{frame}

% ----------------------------------------------------------------------------------
\begin{frame}{4. ADP: (ICU) Admission Probability}

Lag


Distribution:
\begin{equation*}
\text{logit}(\text{ADP}_{j}) \sim \mathcal{N}\left(\text{logit}(\text{madp}_{j}), \text{vadp}_{j}\right)
\end{equation*}

\alert{Among hospitalized patients on day $j$}, let $g_{ij}$ denote the the need for IC of patient $i$
\begin{itemize}
\item $g_{ij}=1$: patient $i$ will require IC at some point (during hospitalization)
\item $g_{ij}=0$: patient $i$ will not require IC
\end{itemize}

For each new hospitalized patient $i$ on day $j$, simulate $g_{ij}$ as
\begin{equation*}
g_{ij} \sim \text{Bernoulli}\left(\text{ADP}_{j}\right)
\end{equation*} 

% ****** Comments
\note{
\tiny
\justifying

}
\end{frame}



% ----------------------------------------------------------------------------------
\begin{frame}{5. LOS: Length of Stay in ICU}

LOS = length of stay (nb days) in ICU (only applies to patients requiring IC)

Distribution:
\begin{equation*}
\text{LOS}_{ij} \sim \text{NegBin}\left(\text{mlos}_{j}, \text{vlos}_{j}\right)
\end{equation*}



\alert{Among hospitalized patients on day $j$}, let $g_{ij}$ denote the the need for IC of patient $i$
\begin{itemize}
\item $g_{ij}=1$: patient $i$ will require IC at some point (during hospitalization)
\item $g_{ij}=0$: patient $i$ will not require IC
\end{itemize}

For each new hospitalized patient $i$ on day $j$, simulate $g_{ij}$ as
\begin{equation*}
g_{ij} \sim \text{Bernoulli}\left(\text{ADP}_{j}\right)
\end{equation*} 

% ****** Comments
\note{
\tiny
\justifying

}
\end{frame}

% ----------------------------------------------------------------------------------
\begin{frame}{Simulations}



% ****** Comments
\note{
\tiny
\justifying

}
\end{frame}

% ##################################################################################
\section{Projections for hospital mortality}
\note{
\tiny
\justifying

}

% ----------------------------------------------------------------------------------
\begin{frame}{Projections for hospital mortality}
\justifying

Uses daily number of new hospitalizations (observed or projected using EGP)

Three age categories ($<$ 70, 70-85, $\geq$ 85): more convenient to change age distribution in the future compared to continuous age


ICP, LAG and total hospital LOS derived using models

We need to simulate patients deaths 
\alert{Competitive risks}




3 models:
\begin{enumerate}
\item Logistic regression for probability to require IC

\item Survival (weibull) fit for time to death

Event: death
Censoring: patients still hospitalized + patients who recovered (i.e. exited hospital alive) 

\item Survival (weibull) fit for time to hospital release (dead or alive)

Event: hospital release (dead or alive)
Censoring: patients still hospitalized

\end{enumerate}


% ****** Comments
\note{
\tiny
\justifying

}
\end{frame}

% ----------------------------------------------------------------------------------
\begin{frame}{Probability to require IC}
\justifying

The probability $p_{i}$ that patient $i$ did require IC at some point during hospitalization was modeled as a function of age and sex in a logistic regression model:

\begin{equation*}
\text{logit}(p_{i})=f(\text{age}_{i}+\text{sex}_{i})
\end{equation*}

IC indicator variable for new patients simulated as $g_{i} \sim \text{Bernoulli}(p_{i})$

For existing patients without IC information (i.e. those still in hospital but not yet admitted in ICU), the indicator $g_{i}$ was imputed  using this model ($M=50$ imputations).

% ****** Comments
\note{
\tiny
\justifying

}
\end{frame}

% ----------------------------------------------------------------------------------
\begin{frame}{Survival models}
\justifying

\alert{Time to death}

Event: death

Censoring: patient still in hospital + patient who left hospital alive (i.e. cured)
\begin{align*}
T_{1i} & \sim \text{Weibull}(\mu_{1i},\sigma_{1}) \\
\log(\mu_{1i}) & = f_{1}(\text{age}_{i}+\text{sex}_{i}+g_{i}) \\
h_{1i} & = \text{instant hazard}
\end{align*}

\alert{Time to hospital release (dead or alive)}

Event: hospital release (dead or alive)

Censoring: patient still in hospital
\begin{align*}
T_{2i} & \sim \text{Weibull}(\mu_{2i},\sigma_{2}) \\
\log(\mu_{2i}) & = f_{2}(\text{age}_{i}*g_{i}+\text{sex}_{i}) \\
h_{2i} & = \text{instant hazard}
\end{align*}

Probability $p^{D}_{i}$ of dying at the end of hospital staygiven by $p^{D}_{i} = h_{1i}/h_{2i}$



% ****** Comments
\note{
\tiny
\justifying

}
\end{frame}

% ----------------------------------------------------------------------------------
\begin{frame}{Simulating patient death (in hospital)}
\justifying

Let $n_{j}$ define the number of existing/new hospitalizations on day $j$
\begin{enumerate}
\item Draw age/sex from $n_{j}$ patients
\item Generate IC status $g_{i}$ according to logistic regression model for $i=1,...,n_{j}$
\item If $g_{i}=1$, generate a lag time according to lag model
\item Generate a total hospital LOS $T_{i}$ using Weibull model for time to hospital release (adjusted for age, sex and $g_{i}$)
\item Calculate hazard of death $h_{1i}=h_{1}(T_{i})$ and hazard of exiting hospital $h_{2i}=h_{2}(T_{i})$
\item Draw death indicator $\delta_{i} \sim \text{Bernoulli}(h_{1i}/h_{2i})$
\end{enumerate}


% ****** Comments
\note{
\tiny
\justifying

}
\end{frame}

% ----------------------------------------------------------------------------------
\begin{frame}{Prediction types for mortality}
\justifying

Three  types of predictions depending on available data:
\pause

\alert{Type 3}
Require aggregated data on hospitalizations (mortality data optional). Uses observed cumulative counts of hospitalized patients and generate new patients/deaths from the start of the epidemic.
\begin{itemize}
\item Prediction intervals for past. Can be compared to observed cumulative death counts (when available) for "goodness of fit"
\item Cumulative deaths counts agree with observed deaths counts
\end{itemize}
\pause

\alert{Type 2}
Require aggregated data on hospitalizations + cumulative death counts. Same as for type 1 except that simulated deaths in the past are replaced with observed death counts
\begin{itemize}
\item point
\end{itemize}
\pause

\alert{Type 1}
Require individual patient data. Uses all individual patient data (i.e. observed age, sex, IC and death status) and generate new patient deaths according to model
\begin{itemize}
\item No prediction intervals for past, 
\item Cumulative deaths counts agree with observed deaths counts
\end{itemize}






% ****** Comments
\note{
\tiny
\justifying

}
\end{frame}

% ----------------------------------------------------------------------------------
\begin{frame}{Limitations}
\justifying




Multi-state models



% ****** Comments
\note{
\tiny
\justifying

}
\end{frame}

\end{document}
